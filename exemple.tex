%-*- coding: utf-8 -*-
\documentclass[a4paper,11pt]{article}
\usepackage[utf8]{inputenc}
\newif\ifrapport

%%%%%%%%%%%%%%%%%%%%%%%%%%%%%%%%%%%%%%%%%%%%%%%%%%%%%%%%%%%%%%%%
% A mettre en commentaire ou non suivant ce qu'on veut imprimer
% \rapporttrue
%%%%%%%%%%%%%%%%%%%%%%%%%%%%%%%%%%%%%%%%%%%%%%%%%%%%%%%%%%%%%%%%
\newcommand{\bla}{\textbf{Oral blanc}}
\newcommand{\blabla}{Semaine du 12 décembre}
\usepackage{colle2}

\begin{document}


%%%%%%%%%%%%%%%%%%%%% SUJET 1 %%%%%%%%%%%%%%%%%%%%%%%%%%%%%%%%%
\begin{colle}

\begin{demo}
    Démontrer le théorème de Pythagore.
\end{demo}


\begin{exercice}
    \renewcommand{\labelitemi}{\textbullet}%
    %Probabilités, formule des probabilités totales
    %série géométrique dérivée
    Monsieur Migros, frontalier embourgeoisé, a décidé de se mettre au footing.
    Il décide d'utiliser un protocole spécial pour savoir combien de kilomètres 
    parcourir chaque jour. Pour ce faire:
    \begin{itemize}
       \item Il lance un dé et compte le nombre $n$ de lancers nécessaires pour 
          obtenir un six.
       \item Il lance ensuite une pièce équilibrée $n$ fois et compte le nombre 
            de «pile» obtenus : c'est le nombre de kilomètres qu'il devra 
        parcourir.
    \end{itemize}
    Dans toute la suite on notera :
    \begin{itemize}
       \item $A_n$:«le premier six est obtenu au $n$-ième lancer»,
       \item $B_p$:«la pièce est tombée $p$ fois sur pile».
    \end{itemize}
    \begin{enumerate}
       \item Calculer $\mathbb{P}(A_n)$ pour tout $n\in\mathbb{N}^\star$
       \item Calculer $\mathbb{P}(B_0)$.
       \item Soit $x\in ]0,1[$. Rappeler la formule permettant de calculer:
            \[ 1+x+x^2+\ldots+x^{n-1}.\]
       \item En dérivant la relation obtenue, déterminer la convergence et la 
            valeur de la série $\sum nx^n$.
       \item En déduire $\mathbb{P}(B_1)$.
    \end{enumerate}

    \begin{correction}
     \begin{enumerate}
      \item On montre que $\mathbb{P}(A_n) = (5/6)^{n-1}\cdot 1/6$.
      \item D'après la formule des probabilités totales,
        pour tout $p\geqslant 0$
      \begin{align*}
        \mathbb{P}(B_p) &= \sum_{\substack{n=p\\n\geqslant 1}}^{+\infty} 
                        \frac{n!}{p!(n-p)!} \left(
                        \frac{5}{12} \right)^{n-1} \frac{1}{12}
        \intertext{et donc}
        \mathbb{P}(B_0) &= \sum_{n=1}^{+\infty} \left(
                        \frac{5}{12} \right)^{n-1} \frac{1}{12} \\
                        &= \frac{1}{7}.
      \end{align*}
      \item facile.
      \item D'après ce qui précède:
        \begin{align*}
          \mathbb{P}(B_1) &= \sum_{n=1}^{+\infty} n \left(
                        \frac{5}{12} \right)^{n-1} \frac{1}{12} \\
                          &= \frac{12}{49}
      \end{align*}
      (vérifier la dernière égalité).
     \end{enumerate}

    \end{correction}
\end{exercice}


\end{colle}


%%%%%%%%%%%%%%%%%%%%% SUJET 2 %%%%%%%%%%%%%%%%%%%%%%%%%%%%%%%%%
\begin{colle}

\begin{exercice}
  Déterminer $A^n$ pour tout $n\in\mathbb{N}$ où:
  \[
    A = \begin{pmatrix}
        -3 & 2 & 1\\
        0 & -2 & 0\\
        -2 & 4 & 0\\
    \end{pmatrix}
  \]
\begin{correction}
On montre que $A$ est diagonalisable. Avec les notations du cours, une solution 
est:
\[ P =\begin{pmatrix}
  1 & -1 & -2\\
  0 & 0 & -1\\
  1 & -2 & 0\\
\end{pmatrix},\, D =\begin{pmatrix}
  -2 & 0 & 0\\
  0 & -1 & 0\\
  0 & 0 & -2\\
\end{pmatrix},\, P^{-1} =\begin{pmatrix}
  2 & -4 & -1\\
  1 & -2 & -1\\
  0 & -1 & 0\\
\end{pmatrix}.\]
\end{correction}

\end{exercice}

\begin{exercice}
     Damien fait du ski à la station ``Vallées blanches''. Il est en
    haut du téléski des Cailloux, et a le choix entre les pistes de
    Tout-Plat (une bleue), Les-Bosses (une rouge) et Rase-Mottes (une
    noire). Il va choisir une de ces trois pistes au hasard, de telle
    façon qu'il choisisse la bleue ou la noire avec probabilité $1/4$,
    et la rouge, qu'il préfère, avec probabilité $1/2$. Il descend
    ensuite la piste choisie. Damien n'est pas encore très à l'aise
    cette saison, et il tombe avec une probabilité de $0,1$ sur la
    piste bleue, de $0,15$ sur la piste rouge, et de $0,4$ sur la
    piste noire.
    \begin{enumerate}
        \item Modéliser la situation.
        \item Soit $A$ l'événement ``\emph{Damien tombe en descendant la piste
            qu'il a choisie}''. Calculer $P(A)$.
        \item Maxime attend Damien à la terrasse d'un
            café et le voit arriver couvert de neige: Damien est tombé.
            Déterminer la probabilité que Damien ait choisi la piste noire.
    \end{enumerate}

\end{exercice}



\end{colle}


\solutions


\end{document}
